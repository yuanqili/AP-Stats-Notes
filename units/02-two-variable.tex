\section{Exploring Two-Variable Data}

\subsection{Unit Guide}

\subsubsection{Developing Understanding}

Building on Unit 1, students will explore relationships in two-variable categorical or quantitative data sets. They will use graphical and numerical methods to investigate an association between two categorical variables. Skills learned while working with two-way tables will transfer to calculating probabilities in Unit 4.

Students will describe form, direction, strength, and unusual features for an association between two quantitative variables. They will assess correlation and, if appropriate, use a linear model to predict values of the response variable from values of the explanatory variable. Students will interpret the least-squares regression line in context, analyze prediction errors (residuals), and explore departures from a linear pattern.

\subsubsection{Building Course Skills}

In Unit 2, students are looking at the relationship between variables. The ability to calculate and describe statistical values, such as a conditional relative frequency or the slope of a regression line, is critical for data analysis because students must be able to analyze patterns before drawing conclusions about the data. Students should be allowed to perform their calculations using technology to help them become more aware of procedural errors. Students will also need practice translating output from technology (``calculator speak'') into appropriate statistical language.

As any statistician will assert, a numerical calculation is only as good as one's ability to interpret what it means in the real world. Rather than just reporting values from their calculations, students must be able to connect their numerical results to the scenario's context and formulate a verbal response that makes that connection clear. Teachers can model good communication and provide high-quality feedback to help students use accurate statistical language when comparing side-by-side bar graphs, for example, and to avoid common errors in reasoning, such as using the word ``line'' to explain why a relationship is linear.

\subsubsection{Preparing for the AP Exam}

Students need ongoing practice with interpretation of vocabulary and calculated values in context. It is typically not sufficient to speak generally about the direction of a relationship, for example. If the question is about a linear model for predicting the weight of a wolf based on its length, students should write that a positive relationship means that longer wolves tend to have higher weights (see 2017 FRQ 1). Students can communicate statistical uncertainty by using words such as ``tend to have'' and ``on average,'' being careful to be precise with language. For example, when explaining evidence of a linear relationship, the difference between discussing a rate of change, as opposed to a change, is the difference between right and wrong. For the sake of clarity, the word ``correlation'' should be reserved for discussions about relationships between two quantitative variables.

\subsection{Least-Squares Regression}

A \textbf{regression line} is a line that describes how a response variable $y$ changes as an explanatory variable $x$ changes. We often use a regression line to predict the value of $y$ for a given value of $x$.

Suppose that $y$ is a response variable and $x$ is an explanatory variable. A regression line relating $y$ to $x$ has an equation of the form
\[
    \hat{y} = a + bx,
\]
where
\begin{itemize}[itemsep=0pt]
\item $\hat{y}$ is the \textbf{predicted value} of the response variable $y$ for a given value of the explanatory variable $x$.
\item $b$ is the \textbf{slope}, the amount by which $y$ is the predicted to change when $x$ increases by one unit.
\item $a$ is the \textbf{$\mathbf{y}$-intercept}, the predicted value of $y$ when $x = 0$.
\end{itemize}

